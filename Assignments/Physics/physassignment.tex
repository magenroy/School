\documentclass[12pt]{article}
\usepackage{setspace}
\usepackage{amsmath}
\doublespacing
\begin{document}
Roy
\\



\indent 1. Counterclockwise 




\indent 2. The angular velocity is equal throughout the record ($\omega$ of point near the center $= \omega$ of point on the edge) since all points are turning the same amount of radians about the center per unit time. The linear velocity of a point increases linearly with its distance from the center since its linear displacement is equal to its arc length which is equal to the product of the radius and the angle, so a point near the edge moves a greater distance per unit time than one near the center (the point near the edge has a greater linear velocity).




\indent 3. 
\begin{gather*}
I_1 = mr^2 \\
I_2 = 3mr^2 \\
I_2 = 3I_1 \\
\\
I_3 = mr_a^2 \text{ where } 2\pi r_a = 2(2\pi r) \\
r_a = 2r \\
I_3 = 4mr^2 \\
I_3 = 4I_1
\end{gather*}




\indent 4. The time it takes to reach the bottom depends on $v_{avg}$

\begin{gather*}
E_t = E_l + E_r + E_g \\
= \frac{I\omega^2 + mv^2 + 2mgh}{2} \\
2E_t - I\omega^2 - 2mgh = mv^2 \\
v = \sqrt{\frac{2E_t - I\omega^2 - 2mgh}{m}} \\
\begin{split}
v_{solid} = \sqrt{\frac{2E_t - mr^2\omega^2 - 2mgh}{m}} \\
\end{split}
\qquad
\begin{split}
v_{hollow} = \sqrt{\frac{2E_t - \frac{mr^2\omega^2}{2} - 2mgh}{m}} \\
\end{split}
\end{gather*}
$v_{hollow}$ is $\frac{r\omega}{\sqrt{2}}$ greater than $v_{solid}$

The hollow cylinder will reach the bottom first



\indent 5.
Before the twizzle, the skater has only linear energy.
During the twizzle, some of the linear energy becomes rotational, the skater slows down.
After the twizzle, the rotational energy becomes linear, the skater speeds up.



\indent 6.
$\theta = \frac{63\pi}{180} = \frac{7\pi}{20} = 1.1$ radians \\
A point on the outer rim moves $r\theta = $ 11cm and $\theta = 1.1$ radians \\
A point at 3cm moves $r\theta = $ 3.3cm and $\theta = 1.1$ radians \\



\indent 7.
\begin{gather*}
\alpha = \frac{\Delta\omega}{\Delta t} \\
= \frac{\omega_2 - \omega_1}{\Delta t} \\
= \frac{1900 - 700}{10} \\
= 120\text{rad/s}
\end{gather*}



8.
\begin{gather*}
\alpha = 32.7\text{rad/s$^2$} \qquad m = 10\text{kg} \qquad r = 50cm\\
\tau = I\alpha \\
= mr^2\alpha \\
= 10(0.5)^232.7 \\
= 81.75 \\
W = \tau\theta \\
= 81.75 \times 33 \times 2\pi \\
= 16950\text{J}
\end{gather*}



9.
\begin{gather*}
E_{g_i} = E_{l_f} + E_{r_f} \\
mgh_i = \frac{mv_f^2 + I\omega_f^2}{2} \\
39.2m = \frac{mv_f^2 + \frac{2 \times 25^2 mr^2}{5}}{2} \\
39.2 = \frac{v_f^2 + 50 \times 0.15^2}{2} \\
78.4 = v^2_f + 1.125 \\
v_f = \sqrt{78.4 - 1.125} \\
= 8.8\text{m/s}
\end{gather*}



10.
\begin{gather*}
S_{front} = S_{back} \\
r_{front}\theta_{front} = r_{back}\theta_{back} \\
1.20 \times (276 \times 2 \pi) = 0.340\theta_{back} \\
\theta_{back} = 6121
\end{gather*}
The back wheel made $6121/2\pi = 974$ revolutions.
\end{document}
