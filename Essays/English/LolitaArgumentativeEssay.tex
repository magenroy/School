\documentclass[12pt]{article}
\usepackage{amsmath}
\usepackage{setspace}
\doublespacing
\begin{document}
\begin{titlepage}
  \noindent Lolita Argumentative Essay \\
  By: Roy Magen \\
  For: Ms. Lefolii \\
  Date: 15/04/14 \\
  Draft: 1
\end{titlepage}
\noindent Vladimir Nabokov's \textit{Lolita} is a postmodern story that draws its value from sources very different from those of soccer. This statement perhaps seems redundant, but sometimes we must be reminded that although Shakespeare's \textit{King Lear} is excellent, we cannot determine the value of all literary works simply by measuring their values against those of \textit{Lear}. Roxena Robinson appears to forget this when referring to \textit{Lolita} in ``The Cold Heart of James Salter.'' The other possible explanation is that she is asserting that all works of literature draw value only from the same places as \textit{King Lear}. This assertion, however; is invalid since it requires that no other work of literature contradict it, but it is being invoked in the context of a work that does exactly that. ``Lacking compassion, interior conflict, and consequences, Nabokov's vision devalues human experience, instead of celebrating it,'' Robinson claims, yet Nabokov's \textit{Lolita} is not devalued because of the hatefulness of some of its characters nor because it explores nontraditional concepts; rather, these are the properties it uses to reach greatness.

Robinson's concept of what makes fiction great contributes to the invalidity of her argument, so let us review this. Roxena Robinson provides two reasons for why we read great fiction, so if we make the harmless assumption that the degree to which a work of fiction is great is proportional to our drive to read it, we can extend this to develop reasons for what makes fiction great. She says that we read great fiction for ``the beauty of the prose'' and ``to expand our understanding of the human heart.'' The first reason does not seem to be problematic. Non-fiction derives much of its value by addressing real phenomena, and so optimizes its prose to be valuable in terms of this property, that is, non-fiction prose is clear. On the other hand, fiction does not benefit as much from being true, instead, fiction literature prioritizes its beauty over its clarity. Roxena Robinson's second reason violates this by prioritizing the exploration of a real thing, the human heart, in fiction. This would require a heavier emphasis on the clarity of the prose than its other properties, such as beauty, which are more suited for fiction. This damages a work of fiction's ability to be great since if it over-prioritizes the achievement of a purpose for which it is not intended, that provides less value, it may fail to achieve an intended purpose and thus fail to reach greatness. Furthermore, Robinson's emphasis on, ``of the human heart,'' also imposes a stylistic problem: why should fiction be limited to such an anthropocentric realm? Even if she had not used the word ``human,'' the result would still be insultingly specific: why should fiction be limited to an exploration of emotions? It may be argued that Robinson did not state that these are the sole reasons for reading great fiction, in which case the aforementioned faults in the second would become irrelevant as they both rely on the principality of the reason, but this argument does not stand given the context in which the second reason is employed. The fact is that Robinson uses the second reason to provide a contrast between \textit{Lolita} and other works that she considers ``great.'' This can be observed both by the use of ``but'' to begin the sentence in which the second reason is stated, indicating contrast to the previous sentence where the first reason appeared that Robinson seems to believe supports even the greatness of \textit{Lolita}: ``Why do we read great fiction ... the beauty of the prose. ... But another reason we read it is to expand our understanding of the human heart ...'' and because the preceding paragraph is composed of a criticism of Nabokov's manner of exploring the human heart, as quoted in the previous paragraph of this essay. So while she did not explicitly specify the principality of her flawed reason, it is implicitly stated through its use her argument. Now that we have established the invalidity of parameters according to which \textit{Lolita} fails to reach greatness, let us establish a new set of parameters and their validity and see that \textit{Lolita} does indeed reach greatness.

Now we will see how \textit{Lolita} conforms quite well to a valid definition of great fiction. Unfortunately, defining a way in which one may judge the value of fiction is particularly difficult. In fact it makes the most sense to argue that the value of something should be defined generally and that then we may categorize things in order to organize them (mostly in order to be able to talk about them). It may seem that this cannot be effective for various reasons, such as that different things have different values in different situations. Yet it only seems this way if we assume that this value takes the form of a single number. Strictly speaking, we are obliged to address this problem from a mathematical perspective in order to be able to reach a decisive answer, but such an approach would likely go well beyond the scope of this essay (this complexity is part of the reason why we do not use mathematics but rather literature in order to explore many concepts), and so we will have to resort to a less rigorous solution for the sake of brevity. In fact, value is a \textit{function} of many things, where things are also not necessarily in the form of plain numbers. Yet this function, when provided with all necessary input, should output a single value. It should be noted that this universal description of a value using a function eliminates the issue of subjectivity by allowing the values that change from person to person to be taken into account in that function. In fact when we categorize ``things,'' we simplify their value functions by inputting some of the variables: the value function for a work of literature requires fewer variables than the value function for a thing, but more variables than the value function for poetry and prose which requires more variables than the value function for fiction and non-fiction, as indicated by the set hierarchy: $\text{thing} \supset \text{literature} \supset (\text{poetry}, \text{prose}) \text{  prose} \supset (\text{fiction prose}, \text{non-fiction prose})$. Now that we have an idea of the scope of value, let us try our best to create a definition for fiction that minimizes loss of generality. One would argue that the degree to which a work of fiction is interesting should be quite important in its value function, even though it applies to far more things than works of fiction. Although it may be argued that in the previous paragraph we criticized the focus of properties not specific to fiction, this is different since the issue above was that a thing that is valued more for one property than another should focus more on maximizing the one property than the other, and this imbalance may be caused not only because this thing is more suited for the one property, but because another thing might be more suited for the other, this is not the case here. In fact, one should also argue that fiction is more suited to be interesting in many cases: non-fiction draws value from its exploration of interesting factual contexts, and fiction predictably draws value from contexts that are interesting but that are not factual. 

Now that we have a working approximation of the value function for fiction, let us see how \textit{Lolita} maximizes it in particular. One aspect of \textit{Lolita} that sets it apart from many other works of fiction is that it goes beyond describing unreal situations, it describes weird and unbelievable situations. Here we see it maximizing in particular the value function of postmodern fiction. Furthermore, the way it gives these description is very interesting.The most obvious difference between \textit{Lolita} and \textit{King Lear} is that the former is a novel and the latter is a play. While this may seem irrelevant as Robinson does not say that one of \textit{Lolita}'s flaws is that it is a novel, it is in fact important since by writing a novel, Nabokov is able to implement various things that are not present and should not be present in \textit{King Lear}. \textit{Lolita} features an unreliable narrator, Humbert Humbert. Humbert Humbert is both the narrator of and a character in \textit{Lotita}. Furthermore, he narrates the story as if the reader is the jury of a trial in which he is participating. This is important and very interesting because it means that we must be constantly aware of all the various biases that might be woven into the narrative. And Humbert Humbert practically reminds us that he is unreliable through all the inconsistencies and sometimes, even more interestingly, through all the consistencies. For example, Vivian Darkbloom is an anagram of the authors, wrote the play ``The Enchanted Hunters'' in which Lolita preforms. This is also the name of a hotel. This name is sometimes switched to ``Hunted Enchanters.'' In the library of the jail in which Humbert Humbert is writing the book, there is a book that mention Vivian Darkbloom, and Clare Quilty, as well as ``The Little Nymph'' which corresponds to nymphets and the ``The Lady who Loved Lightning'' which corresponds to the way in which Humbert's mother died. \textit{Lolita} is full of these massively interesting and massive coincidence nets and is only a part of what make \textit{Lolita} such an incredibly interesting work of fiction.

Both \textit{Lolita} and \textit{King Lear} are great works of fiction, yet they are great in remarkably different ways. \textit{King Lear} is great because of the emotions that it inspires and because of its interesting exploration of interesting concepts such as compassion. The fact that another work of literature strives for greatness not by replicating the virtues of \textit{King Lear} does not constitute its fault. In fact, it is more interesting to explore less common qualities than those already thoroughly mapped. And so Vladimir Nabokov ventures into relatively unexplored waters, and explores a work of fiction's ability to be interesting by including a massive variety of interesting games in his book \textit{Lolita}, things ranging from puns to an unreliable narrator. We must be careful not to let great literature narrow our willingness to appreciate other instances of it: the excellency of \textit{King Lear} does not prevent that of \textit{Lolita}.

\newpage
Literature Cited: \newline
Robinson, Roxena. ``The Cold Heart of James Salter''
\end{document}
