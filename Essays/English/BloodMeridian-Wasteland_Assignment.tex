\documentclass[12pt]{article}
\linespread{2}
\begin{document}

	%Titlepage
	\begin{titlepage}
		\noindent \uline{Blood Meridian} \uline{The Wasteland} English Essay \\
		By: Roy Magen \\
		For:  Ms. Lefolii \\
		Date: 31/04/2014 \\
		Draft: 1
	\end{titlepage}

	\paragraph{3)}
		Both the story of the Holy Grail and \uline{The Waste Land}
		take place in a wasteland. Both wastelands were once great, but
		have collapsed.	
	
	\paragraph{5)}
		\uline{Blood Meridian} makes various biblical references,
		particularly to the \uline{Book of Revelations}.

	\paragraph{6)}
		Both \uline{The Waste Land} and \uline{Blood Meridian} discuss
		dystopic settings. The contrast of the language used with the
		medium through which it is expressed serves to renew the
		meaningfulness of the words [mention Virginia Wolf].
		Furthermore, \uline{The Waste Land} aims to maximize first
		its intellectual value, as can be seen by the quantity of
		allusions, before its raw aesthetic value, which would be
		provided by more poetic language. 

\end{document}
